\documentclass[UTF8]{ctexbook}

\usepackage[titles]{tocloft} % 目录
\usepackage[colorlinks,linkcolor=blue]{hyperref}
% \usepackage{graphicx} % An example of a floating figure using the graphicx package.
\usepackage{animate}
\usepackage{enumerate}

\renewcommand\thesection{\arabic{section}}
% arabic 阿拉伯数字 roman 小写的罗马数字 Roman 大写的罗马数字 alph 小写字母  Alph 大写字母
\begin{document}
% ----------- 封面 ---------------
\begin{center}  % 居中
	\quad \\
	\vspace{3cm}
	\hspace{3cm}\Huge{A NOTE of MASTER}\\
	\hspace{3cm}\Huge{菜鸟硕士手册} \\
	\hspace{3cm}\large{V1.0} \\
	\vspace{5cm}
	\hspace{3cm}\href{https://github.com/lonelybag/Latex_lonelybag}{\Large{@LonelyBag}}\\
	\vspace{0.5cm}
	\hspace{3cm}\Large {2018.3.2}
	\clearpage  % 清除当页页码
\end{center}

\thispagestyle{empty}
\clearpage  % 清除当页页码
% ----------- 目录 ---------------
\tableofcontents
\clearpage
\newpage
\quad
\newpage
\clearpage
% ----------- 正文 ---------------
\section{快速抓住研究热点}
研究热度往往体现在论文的发表数量中,但是一个领域中的主要研究方向却往往很难仅通过阅读论文而获得全面认识。换句话说,统计论文中的研究方向远比统计发表数量重要。为了解决这一问题,研究人员也提出了若干解决方案。

\subsection{传统方式}
词典,是语言学家针对特定语言而将常见字词按照一定的逻辑编纂起来的文本工具。科学家也有自己的词典,他们针对某一特定研究领域对热点论文进行总结归纳并提出相应见解,最终以论文的形式发表,这就是文献综述。

通过阅读文献综述,可以快速地对所关注的研究领域形成大致了解,而这也是快速抓住研究热点的传统方式。通过限定检索词可以较为有效地检索该类型的文章,对于英文文献,其检索词可以选择如下几种:
\begin{itemize}
	\item Review
	\item Challenge
	\item Survey
	\item Statement
	\item ... ...
\end{itemize}

这样,再加上一些限定词(如:WPT)就可以有效地检索出特定研究领域的文献综述。但是,这样检索出的文献往往比较松散,并且会使我们忽略一些更有价值的paper。因此,有时(甚至往往)会采用辅助工具进行这项工作。

\subsection{辅助工具}
为了快速地了解一个领域的研究热点,仅仅通过 {\bf 关键词检索 - 论文下载 - 阅读 - 二次检索} 这类流程是不能达到“快速”的要求的,真正有效的方法是借助论文分析工具,这类工具一般有如下特点:

\begin{itemize}
	\item 可以分析大量文献间的交叉检索关系
	\item 结果可视化
	\item 多功能
\end{itemize}

常用的辅助工具有:\href{https://zhuanlan.zhihu.com/p/20902898}{HistCite Pro 2.1} 以及 \href{https://zhuanlan.zhihu.com/p/30970993}{VOSviewer}。这两种工具的最大区别就是HistCite Pro 2.1仅支持英文文献,但VOSviewer不仅支持英文还支持中国知网导出的中文文献。笔者只用过HistCite Pro 2.1,如果读者希望对中文文献也进行处理,可以参考\href{https://www.jianshu.com/p/e20f3f1d17d8}{这个})。下面我通过若干gif对软件的使用进行介绍。
\subsubsection{安装}
安装就不赘述了,参考\href{https://zhuanlan.zhihu.com/p/20902898}{这里}就够了。
\subsubsection{导出文献}
这一步是初学者最容易出现差错的地方,这是因为HistCite Pro 2.1对源数据的格式要求非常严谨,而且它仅支持由Web of Science导出的文献格式。下面我通过Step-by-step的方式展示如何从WOS导出符合要求的格式。

第一步:选择检索{\bf 数据库}并给定检索词,一般首次使用需要我们对整个领域有一个宏观认识,因此检索词可以非常概括,比如我使用的wireless power,如图\ref{fig:1}。

\begin{figure}[!htb]
	\centering
	\animategraphics[autoplay, loop , width=0.9\linewidth]{8}{Figure//WOSsearch//10}{01}{80}
	\vspace{-0.3cm}
	\caption{WOS检索关键词:wireless power}\label{fig:1}
\end{figure}

第二步:选择正确格式,下载,如图\ref{fig:2}所示。值得注意的是,WOS仅支持一次性导出500篇文献,所以更多的文献需要多次下载。

\begin{figure}[!htb]
	\centering
	\animategraphics[autoplay, loop , width=0.9\linewidth]{8}{Figure//WOSdownload//00}{01}{98}
	\vspace{-0.3cm}
	\caption{WOS下载特定格式检索结果}\label{fig:2}
\end{figure}

第三步,需要将下载获得的所有txt移动至TXT文件夹,然后打开main.exe,此时输入1并回车,软件将自动打开程序界面,接着点击gragh maker,修改count将对分析的文献数量进行控制,一般来说,取得大一点可以获得更多的信息。注意到,select by一栏有两个选项:LCS和GCS,LCS是Local Citation Score的缩写,代表某篇文献在本地数据集(也就是我们所下载的txt集合)的总被引次数,而GCS是Global Citation Score的缩写,代表在WOS数据库中的总被引次数。一般来说,由于本地数据集是根据我们的关键词获得的,因此LCS排名更能够反映某一学科的发展。

\begin{figure}[!htb]
	\centering
	\animategraphics[autoplay, loop , width=0.9\linewidth]{8}{Figure//Hmaker//00}{01}{57}
	\vspace{-0.3cm}
	\caption{WOS下载特定格式检索结果}\label{fig:3}
\end{figure}

\section{常用网站}


\section{COMSOL}

\begin{figure}[!htb]
	\centering
	\animategraphics[controls, autoplay, loop , width=0.6\linewidth]{8}{Figure//Fig2//10}{01}{19}
	\vspace{-0.3cm}
	\caption{This is an awesome pdf with a gif - by lonelybag.}\label{fig:39}
\end{figure}

\section{文献写作}
\subsection{写作工具}
\subsection{写作流程}
\subsection{参考文献}

\end{document}
