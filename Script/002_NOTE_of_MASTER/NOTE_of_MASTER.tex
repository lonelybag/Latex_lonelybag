\documentclass[UTF8]{ctexbook}

\usepackage[titles]{tocloft} % 目录
% \usepackage{graphicx} % An example of a floating figure using the graphicx package.
\usepackage{animate}
\usepackage{enumerate}

\renewcommand\thesection{\arabic{section}}
% arabic 阿拉伯数字 roman 小写的罗马数字 Roman 大写的罗马数字 alph 小写字母  Alph 大写字母
\begin{document}
% ----------- 封面 ---------------
\begin{center}  % 居中
	\quad \\
	\vspace{3cm}
	\hspace{3cm}\Huge{A NOTE of MASTER}\\
	\Huge{我是一个皮儿} \\
	\vspace{5cm}
	\hspace{3cm}\Large{@LonelyBag}\\
	\vspace{0.5cm}
	\hspace{3cm}\Large {2018.3.2}
	\clearpage  % 清除当页页码
\end{center}

\thispagestyle{empty}
\clearpage  % 清除当页页码
% ----------- 目录 ---------------
\tableofcontents
\newpage
% ----------- 正文 ---------------
\section{快速抓住热点}
研究热度往往体现在论文的发表数量中,但是一个领域中的主要研究方向却往往很难仅通过阅读论文而获得全面认识。换句话说,统计论文中的研究方向远比统计发表数量重要。为了解决这一问题,研究人员也提出了若干解决方案。

\subsection{传统方式}
词典,是语言学家针对特定语言而将常见字词按照一定的逻辑编纂起来的文本工具。科学家也有自己的词典,他们针对某一特定研究领域对热点论文进行总结归纳并提出相应见解,最终以论文的形式发表,这就是文献综述。

通过阅读文献综述,可以快速地对所关注的研究领域形成大致了解,而这也是快速抓住研究热点的传统方式。

\subsection{辅助工具}
为了快速地了解一个领域的研究热点,仅仅通过 {\bf 关键词检索 - 论文下载 - 阅读 - 二次检索} 这类流程是不能达到“快速”的要求的,真正有效的方法是借助论文分析工具,这类工具一般有如下特点:

\begin{itemize}
	\item 可以分析大量文献间的交叉检索关系
	\item 结果可视化
	\item 多功能
\end{itemize}

\section{COMSOL}
\begin{figure}[!htb]
	\centering
	\animategraphics[controls, autoplay, loop , width=0.6\linewidth]{8}{Figure//Fig2//10}{01}{19}
	\vspace{-0.3cm}
	\caption{This is an awesome pdf with a gif - by lonelybag.}\label{fig:1}
\end{figure}
\subsection{常用网站}

\end{document}