\documentclass[UTF8,oneside]{ctexbook}

% \usepackage[utf8]{inputenc} %指定编码方式为utf8
\usepackage[titles]{tocloft} % 目录
% \usepackage{fancyhdr} % 页眉页脚
% \pagestyle{fancy} % 美化页眉
%\usepackage{appendix}
\usepackage{graphicx} % An example of a floating figure using the graphicx package.
\usepackage{animate} % 动画
\usepackage[caption=true, font=footnotesize]{subfig}% 前面的[]为了防止覆盖IEEE默认选项
\usepackage{float} %禁止浮动
\usepackage{enumerate} % 列表
\usepackage{xpinyin} % 注音
\usepackage{amsmath} % 数学
\usepackage{listings} %抄录环境
\usepackage{color} %定义颜色
\definecolor{codegreen}{rgb}{0,0.6,0}
\definecolor{codegray}{rgb}{0.5,0.5,0.5}
\definecolor{codepurple}{rgb}{0.58,0,0.82}
\definecolor{backcolour}{rgb}{0.95,0.95,0.92}


\usepackage[colorlinks,linkcolor=blue]{hyperref} % 超链接
\usepackage[margin=3cm]{geometry} % 更改页边距


\title{{\LaTeX 宏包、指令与结果}\\
\large{V1.0}}
\author{\href{https://github.com/lonelybag?tab=repositories}{\itshape{@LonelyBag}}\\ {\itshape \large{Edite by \href{https://mirrors.tuna.tsinghua.edu.cn/CTAN/systems/texlive/Images/}{\LaTeX}}}}

\begin{document}
% ----------- 封面 ---------------
\frontmatter
\maketitle
% ----------- 目录 ---------------
\tableofcontents
\mainmatter
%--------- 定义代码抄录格式 -------
\lstset{
	backgroundcolor=\color{backcolour},
	commentstyle=\color{codegreen},
	keywordstyle=\color{magenta},
	numberstyle=\tiny\color{codegray},
	stringstyle=\color{codepurple},
	basicstyle=\footnotesize,
	breakatwhitespace=false,
	breaklines=true, % 允许断行                 
	captionpos=b,
	keepspaces=true,
	numbers=left,
	numbersep=5pt,
	showspaces=false,
	showstringspaces=false,
	showtabs=false,
	tabsize=2
}

% ----------- 正文 ---------------
\chapter{环境}
% \iffalse \section{========= 不编译 开 =========}
% \section{========= 不编译 关 =========}\fi
% \section{========= 结束 =========}\end{document}
\section{英文环境}


\section{中文环境}

\section{常用指令}
\subsection{超链接}
\begin{lstlisting}[language=Matlab,frame=single]
\usepackage{hyperref}
\href{https://github.com/lonelybag/Latex_lonelybag}{Lonelybag的GitHub}
\end{lstlisting}
编译结果:\\
\href{https://github.com/lonelybag/Latex_lonelybag}{Lonelybag的GitHub}

\subsection{标尺盒子}
\begin{lstlisting}[language=Matlab,frame=single]
% 无宏包
% 用法:\rule[raise]{width}{thickness}

\begin{tabular}{|c|}
    \hline
    \rule[-1em]{1em}{1ex}下沉当前字号下M的宽度 \\
    \hline
    \rule[-7pt]{1pt}{28pt}下沉7磅       \\
    \hline
    无盒子                             \\
\end{tabular}
\end{lstlisting}
编译结果:\\
\begin{tabular}{|c|}
    \hline
    \rule[-1em]{1em}{1ex}下沉当前字号下M的宽度 \\
    \hline
    下沉7磅\rule[-7pt]{1pt}{28pt}   \\
    \hline
    无盒子                        \\
    \hline
\end{tabular}

\subsection{抄录}
\begin{lstlisting}[language=Matlab,frame=single]
\usepackage{listings} %抄录环境
% 必须预先定义抄录格式
\lstset{
	backgroundcolor=\color{backcolour},
	commentstyle=\color{codegreen},
	keywordstyle=\color{magenta},
	numberstyle=\tiny\color{codegray},
	stringstyle=\color{codepurple},
	basicstyle=\footnotesize,
	breakatwhitespace=false,
	breaklines=true, % 允许断行                 
	captionpos=b,
	keepspaces=true,
	numbers=none,
	numbersep=5pt,
	showspaces=false,
	showstringspaces=false,
	showtabs=false,
	tabsize=2
}
% 脚本
\begin{lstlisting}[language=Matlab,frame=single]
这是抄录内容
end{lstlisting}   
\end{lstlisting}
编译结果:\\
\begin{lstlisting}[language=Matlab,frame=single]
这是抄录内容
\end{lstlisting}

\subsection{动画}
\begin{lstlisting}[language=Matlab,frame=single]
\usepackage{animate} % 动画
\usepackage{float} % 禁止浮动,可将figure参数设置为H,否则只能是hbt

\begin{figure}[H]
    \centering
    \animategraphics[autoplay, loop , width=0.4\linewidth]{10}{Pics//workspace//}{1}{30}
    \vspace{-0.3cm}
    \caption{command window - base workspace}\label{fig:1}
\end{figure}
\end{lstlisting}
编译结果:\\
\end{document}